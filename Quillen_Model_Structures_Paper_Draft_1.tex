\documentclass{amsart}
\usepackage{mmap} % make PDF files generated by pdfLaTeX both searchable and copy-able in acrobat reader and other compliant PDF viewers
\usepackage{amssymb}
\usepackage{graphicx}
\usepackage{hyperref}
\usepackage{enumerate}

\usepackage{amssymb}
%================================ AMSTHM ================================
\swapnumbers
\newtheorem{thm}{Theorem}[section]
%\newtheorem{theorem}{Theorem}
\newtheorem{conjecture}[thm]{Conjecture}
\newtheorem{lem}[thm]{Lemma}
%\newtheorem{lemma}[theorem]{Lemma}
\newtheorem{cor}[thm]{Corollary}
%\newtheorem{corollary}[theorem]{Corollary}
\newtheorem{prop}[thm]{Proposition}
%\newtheorem{proposition}[theorem]{Proposition}
%\newtheorem{definition}[theorem]{Definition}
%\newtheorem{example}[theorem]{Example}
%\newtheorem{exercise}[thm]{Exercise}
%\newtheorem{exercise}[theorem]{Exercise}
\newtheorem{claim}[thm]{Claim}
\newtheorem{law}[thm]{Law}
\newtheorem*{thm*}{Theorem}
\newtheorem*{lem*}{Lemma}
\newtheorem*{conjecture*}{Conjecture}
\newtheorem*{cor*}{Corollary}
\newtheorem*{prop*}{Proposition}
\newtheorem*{exercise*}{Exercise}
\newtheorem*{law*}{Law}
\newtheorem*{claim*}{Claim}


\theoremstyle{definition} \newtheorem{defn}[thm]{Definition}
\theoremstyle{definition} \newtheorem*{defn*}{Definition}
\theoremstyle{definition} \newtheorem{xca}[thm]{Exercise}
\theoremstyle{definition} \newtheorem{remark}[thm]{Remark}
\newtheorem{example}[thm]{Example}
\newtheorem*{example*}{Example}
\newtheorem{eg}[thm]{Example}
\newtheorem*{eg*}{Example}

\newtheorem{fact}[thm]{Fact}
\newtheorem*{fact*}{Fact}

\newcommand{\thmref}[1]{Theorem~\ref{#1}}
%============================== End AMSTHM ==============================

\begin{document}
\title{A gentle introduction to a model category of topological spaces}
\author[J. Gross]{Jason Gross}
\address{Massachusetts Institute of Technology}
\email{\href{mailto:jgross@mit.edu}{jgross@mit.edu}}
\date{\TeX ed on \today}
%\subjclass[2000]{Primary 18B30; Secondary 54-01}
%Primary: Categories of topological spaces and continuous mappings
%Secondary: General topology -- Instructional exposition (textbooks, tutorial papers, etc.)
%\thanks{}
%\keywords{Quillen,model category,topology,model structure}

\begin{abstract}
  FIX
\end{abstract}

\maketitle

\section{Introduction}

\section{Category Theory}
  \begin{defn}[Category]
  \end{defn}
  \begin{defn}[Commutative Diagram]
  \end{defn}
  \begin{defn}[Functor]
  \end{defn}
  \begin{defn}[Natural Transformation]
  \end{defn}
  \begin{defn}[Natural Equivalence]
  \end{defn}  
  \begin{defn}[Opposite Category]
  \end{defn}
  \begin{defn}[Retract] % MC3
  \end{defn}
  \begin{defn}[Colimit] % MC1
  \end{defn}
  \begin{defn}[Coproduct]
  \end{defn}
  \begin{defn}[Pushout]
  \end{defn}
  \begin{defn}[Limits] % MC1
  \end{defn}
  \begin{defn}[Product]
  \end{defn}
  \begin{defn}[Pullback]
  \end{defn}

\section{Topological Background}
  \subsection{Weak Homotopy Equivalences}
  \subsection{Hurewicz Fibrations}
  \subsection{Closed Hurewicz Cofibrations}
  
\section{Model Categories}
  \begin{defn}[Lift]
  \end{defn}
  \subsection{Axioms}

\section{Homotopy}
  \subsection{Cylinder Objects}
  \subsection{Left Homotopy}
  \subsection{Path Object}
  \subsection{Right Homotopy}
  \subsection{Homotopic Maps}
  
\subsection{Further Reading}
  
  \cite{sets_maps_limits_colimits}
  \cite{dwyer1995homotopy}
\nocite{*}
\bibliographystyle{plain}
\bibliography{quillen_model_structures}
\end{document}